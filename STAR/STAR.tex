\documentclass{article}
\usepackage[utf8]{inputenc}

\usepackage{authblk}
\usepackage{multicol}
%\usepackage{natbib}
\usepackage{graphicx}
\usepackage{abstract}
\usepackage{float}
\usepackage[english]{babel}
\usepackage{amsmath}
\usepackage{subfigure}
\usepackage{alltt}
\usepackage{url}
\usepackage{multirow}
\usepackage[margin=4cm]{geometry}



\title{Human Connectome: visualization techniques and methodologies}
\author{Giorgio Conte}
%\date{December, $11^{th}$ 2014}
\affil{Creative Coding Lab\\ Department of Computer Science\\University of Illinois at Chicago}



\begin{document}

\maketitle
\begin{abstract}
\end{abstract}

\begin{multicols}{2}
\raggedcolumns

\section{Introduction}

Being able to deeply understand how the brain is working is one of the main challenges in the last years among neuroscientists. With the advent and the refinement of new technologies like fMRi and diffusion MRI, doctors are able to collect and derive data about how the regions of the brain are connected. Very frequently the map of neural connections of the brain is addressed as CONNECTOME.
Visualizing these data in an effective way allows people to navigate and explore all the wired connections that are in the brain. Moreover, thanks to this kind of visualizations is possible to understand better what differences there are between healthy subjects and other people who suffer from a wide range of neuropsychiatric illnesses like bipolar, body dysmorphic disorder, Alzheimer?s disease and late-life depression. 
Many visualization tools have been proposed in the academic literature, however the vast majority of them perform 2D visualizations. However, since this research field is quite novel, there is still room for improvement.
The aim of this work is to report and survey the visualizations tools already present in the academic literature as well as to introduce which the new trends for the near future are.\\
The paper is structured as follows. In section there is a more detailed introduction about the human connectome. Then, a detailed list of common tasks follows, while in section NUMERO I describe accurately the most interesting tool. In section ... to be continued.
%-------------------------------------------------------------------------
\section{Domain}
Human Connectome has been always considered by neuroscientists a very interesting and challenging topic. However, it is only in the last few years, when more powerful and more accurate technologies took place firmly in the research area, that more detailed studies have been conducted. In fact, thanks to very advanced techniques and algorithms, it is now possible to collect data about the functional, structural and anatomical connectivity of the brain. Then, the starting points of this investigation are the data collected from fMRI and diffusion MRI. What the neuroscientists can obtain is a parcellation of the brain into smaller subregions as well as the strength of the connections, whether structural or functional, between the regions. So, the entire Connectome could be reduced to a very dense and highly connected graph. Rubinov and Sporns in \cite{complexNetworkMeasures} were the first authors that consider the human connector as a graph and in turns they described many graph-based metrics applied to the connectome.  Since as aforementioned the networks represented are highly dense, the main challenge should be addressed is that task of "creating intuitive, informative and candid images" highlighted by Margulies et al. in \cite{visualizingHumanConnectome}

\section{Main Tasks}
One of the worse mistake that can be done when designing a visual analytics tool is the willingness of addressing a spread set of tasks. So that is why is important to clearly understand and then state what the main tasks in a given research area are. This activity it is not as simple as it might seem at a first glance. So, that is why in this section I am trying to describe as much clear as possible the main goals that neuroscientist would achieve using a human connectome visualization tool.\\
Due to the high complexity of the brain network, \textit{exploration} is the the most important task. Although some of readers may argue that the exploration task is too simple, pretty obvious and too general, in my opinion, it is not so. In fact, especially when the field is quite novel and with many uncleared aspects, as it is the one we are talking about, it is extremely relevant to allow user a visualization flexibility. Flexibility in terms of level of abstraction, 


\section{Visualization Tools}
Many visualization tools have been presented in the literature. Before describing them more in details I want to report the interesting taxonomy introduced by Marguilies et al in \cite{visualizingHumanConnectome}. In fact, they identified three main categories 

\section{Future Works and new trends}


\bibliography{mybib}{}
\bibliographystyle{plain}  




%------------------------------------------------

\end{multicols}
\end{document}